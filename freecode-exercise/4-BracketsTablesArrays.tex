\documentclass[18pt]{article}    % Document class with font size 18pt
\usepackage{amsfonts,amssymb,amsmath,float}   % Math and font packages, and float to force table positions

%close indent
\parindent 0px                  % No indentation at the beginning of paragraphs
\pagestyle{empty}              % No page number or header/footer

\begin{document}

Small Bracket :

The Distributive Property states that $a(b+c)=ab+ac$,for all $a,b,c \in \mathbb{R}$    % Inline math using ()

Square Bracket :

the equivalence class of $a$ is $[a]$.       % Inline math using []

Curly Brackets :

The set $A$ is defined to be $\{1,2,3\}$.    % Inline math using {}

Dollar sign :

The movie ticket cost $\$11.50$.            % Escaping $ with \ to print dollar sign

Use Case of left and right

in small bracket
$$  2\left(\frac{1}{x^2-1}\right)  $$     % Using \left( and \right) to auto-scale parentheses

in curly bracket
$$  2\left\{\frac{1}{x^2-1}\right\}  $$   % Using \left\{ and \right\} for curly braces

in square bracket
$$ 2\left [\frac{1}{x^2-1}\right]  $$     % Using \left[ and \right] for square brackets

in <> bracket 
$$  2\left<\frac{1}{x^2-1}\right>  $$     % Angle brackets (not recommended, use langle/rangle instead)

or using rangle and langle

$$  2\left \langle \frac{1}{x^2-1} \right \rangle    $$   % Correct way to use angle brackets

$$  \left. \frac{dy}{dx} \right |_{x=1}  $$   % Use of \left. and \right| to format evaluation at x=1

complex fraction

$$ \left (\frac{1}{1 + \left(\frac{1}{1+x}\right)}\right)  $$   % Nested fractions with scaling parentheses

table : 

\begin{tabular}{|c||c|c|c|c|c|}    % Table with vertical lines, double line between first and others
% c , l , r can be used with pipe to distuinguish between
% use || to distuinguish the first coloum with other
\hline
$x$ & 1 & 2 & 3 & 4 & 5 \\ \hline
$f(x)$ & 10 & 11 & 12 & 13 & 14 \\ \hline
\end{tabular}

\vspace{1cm}      % Adds vertical space between elements

\begin{table}[H]       % Float positioning to force table exactly here
\centering
\def\arraystretch{1.8}   % Increase row height
\begin{tabular}{|c||c|c|c|c|c|}    
% c , l , r can be used with pipe to distuinguish between
% use || to distuinguish the first coloum with other
\hline
$x$ & 1 & 2 & 3 & 4 & 5 \\ \hline
$f(x)$ & $\frac{1}{2}$ & 11 & 12 & 13 & 14 \\ \hline
\end{tabular}
\caption{These values represent the function of x}  % Caption for table
\end{table}

\begin{table}[H]
\centering
\def\arraystretch{1.8}
\begin{tabular}{|l|p{2in}|}    % First column left aligned, second column has fixed width (paragraph type)
% c , l , r can be used with pipe to distuinguish between
% use || to distuinguish the first coloum with other
\hline
$f(x)$ & $f'(x)$ \\ \hline
$x > 0$ & The function $f(x)$ is increasing. \\ \hline
\end{tabular}
\caption{the relationship between the $f$ and $f'$}
\end{table}

Arrays : 

\begin{align}				% Align environment for equations with numbering
5x^2    \text{place your words here}   \\      % Text inside equation using \text{}
5x^2-9 = x+3 \\
5x^2-x-12 = 0
\end{align}

\begin{align}				
5x^2    \text{place your words here}   \\
5x^2-9 &= x+3 \\				% use & before = sign to align equations to =
5x^2-x-12 &= 0
\end{align}

% if you dont want numbered equation then use align*
\begin{align*}				% Align* suppresses equation numbers
5x^2    \text{place your words here}   \\
5x^2-9 &= x+3 \\				
5x^2-x-12 &= 0
\end{align*}

\end{document}
