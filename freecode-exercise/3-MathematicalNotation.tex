\documentclass[18]{article}              % Document class with font size option (though 18 is not standard)
\pagestyle{empty}                        % No page numbering or headers/footers
\usepackage{amsmath,amssymb,amsfonts}    % Useful math and symbol packages

\begin{document}

% superscripts
superscripts $${2x^3}$$                  % Superscript: x raised to the power 3

% use curly bracket to use double or more digit in superset
$${2x^{34}}$$                            % Use curly braces to group entire exponent
$${2x^{3x+4}}$$                          % Exponent with expression
$${2^{3x^4+5}}$$                         % Superscript nested with another superscript

Subscript
$$x_1$$                                  % Subscript: x with subscript 1

% use curly bracket for more than 2 or two subscript
$${x_{123}}$$                            % Subscript with multiple digits using braces
$${x_{1_2}}$$                            % Nested subscript: 1 subscripted by 2
$${x_{1_{2_3}}}$$                        % Deeply nested subscripts

lower dots
$${a_0,a_1,a_3,\ldots,a_{100}}$$        % \ldots: ellipsis aligned with baseline (used in subscripts or lists)

centre dots
$${a_0,a_1,a_3,\cdots,a_{100}}$$        % \cdots: ellipsis centered vertically (used in products/sequences)

Greek Letters
$$\alpha$$                              % Lowercase Greek letter alpha
$$A=\pi r^2$$                           % Area of circle using Greek letter pi

Trig Functions
$$y=\sin x$$                            % Sine function
$$y=\cos x$$                            % Cosine function
$$y=\csc \theta$$                       % Cosecant function with theta
$$y=\sin^{-1}x $$                       % Inverse sine (notation)
$$y=\arcsin x$$                         % Alternative notation for inverse sine

Log Function
$$y=\log x$$                            % Common logarithm
$$y=\log_{10} x$$                       % Logarithm with base 10
$$y=\ln x$$                             % Natural logarithm

Roots
$$\sqrt{2}$$                            % Square root of 2
$$\sqrt[3]{9}$$                         % Cube root of 9
$$\sqrt{x^2+y^2}$$                      % Root of expression
$$\sqrt{ 1+\sqrt{x}  }$$                % Nested square roots

Fractions
$$\frac{3}{5}$$                         % Basic fraction

Without Displaystyle :About $\frac{2}{3}$ of the Glass is full.\\[11pt]     % Inline math (small fraction size)

With Displaystyle :About $\displaystyle \frac{2}{3}$ of the Glass is full   % Forces large fraction in inline

With dfrac : About $\dfrac{2}{3}$ of the Glass is full[11pt]                % \dfrac from amsmath (forces large display style)

$$\frac{\sqrt{x+1}}{\sqrt{x+2}}$$       % Fraction with square roots

$$\frac{1}{1+\frac{1}{\frac{1}{x}}}$$   % Complex nested fractions

\end{document}
